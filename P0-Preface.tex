% !TeX root = P0-Intro.tex 
\documentclass[Book-Poly]{subfiles}
\begin{document}


%------------ Chapter ------------%
\chapter*{Preface}\label{chapter.0}

\begin{quote}
	The proposal is also intended to [serve] equally as a foundation for the academic, intellectual, and technological, on the one hand, and for the curious, the moral, the erotic, the political, the artistic, and the sheerly obstreperous, on the other.\\
\mbox{}\hfill ---Brian Cantwell Smith\\
\mbox{}\hfill \emph{On the Origin of Objects}
\end{quote}

\begin{quote}
For me, though, it is difficult to resist the idea that space-time is not essentially different from matter, which we understand more deeply. If so, it will consist of vast numbers of identical units---``particles of space''---each in contact with a few neighbors, exchanging messages, joining and breaking apart, giving birth and passing away.\\
\mbox{}\hfill ---Frank Wilczek\\
\mbox{}\hfill \emph{Fundamentals}
\end{quote}

During the Fifth International Conference on Applied Category Theory in 2022, at least twelve of the fifty-nine presentations and two of the ten posters referenced the category of polynomial functors and dependent lenses or its close cousins (categories of optics and Dialectica categories) and the way they model diverse forms of interactive behavior.
At the same time, all that is needed to grasp the construction of this category---called $\poly$ for short---is an understanding of mathematical sets and functions.
There is no need for the theory and applications of polynomial functors to remain the stuff of technical papers; $\poly$ is far too versatile, too full of potential, to be kept out of reach.

\section*{Purpose and prerequisites}

A categorical theory of general interaction must be interdisciplinary by its very nature.
Already, drafts of this text have been read by everyone from algebraic geometers to neuroscientists and AI developers.
We hope to extend our reach ever further, to bring together thinkers and tinkerers from a diverse array of backgrounds under a common language by which to study interactive systems categorically.
In short---we know about $\poly$; you know about other things; but only our collective knowledge can reveal how $\poly$ could be applied to those other things.

As such, we have strived to write a friendly and accessible expository text that can serve as a stepping stone toward further investigations into polynomial functors.
We include exercises and, crucially, solutions to guide the learning process; we draw extensive analogies to provide motivation and develop intuition; we pose examples whenever necessary.
Proofs may bear far more detail than you would find in a research paper, but not so much detail that it would clutter the key ideas.

On the other hand, there is some deep mathematical substance to the work we will discuss, drawing from the well-established theory of categories.
Although you will find, for example, a complete proof of the Yoneda lemma within these pages, we don't intend to build up everything from scratch.
There are plenty of excellent resources for learning category theory out there, catering to a variety of needs, without adding another to the mix that is really trying to introduce $\poly$.
So for the sake of contributing only what is genuinely helpful, we assume a certain level of mathematical background---a good rule of thumb is if you can define the following fundamental concepts from category theory, and give examples of each:
\begin{itemize}
    \item categories,
    \item functors,
    \item natural transformations,
    \item (co)limits,
    \item adjunctions, and
    \item (symmetric) monoidal categories.
\end{itemize}



\section*{Choices and conventions}

Some key proofs are even depicted in pictures, yet we contend that they are no less rigorous than the cloud of notation whose place they take.



\section*{Past, present, and future}

DJM visited DS in Boston for almost all of 2020. Mutual excitement over different but related things. DS super excited about $\poly$, DJM super excited about doctrines and later paradigms.

DJM encodes something like Poly in Idris and teaches Idris to DS, who having just learned Haskell (thanks Bartosz!) quickly ports the ideas to the strict $\poly$ setting. We discussed writing a book together---a $T$-shape, consisting of a broad generalization with paradigms and a deep dive into $\poly$---after ACT2020, in July.

DS realized comonoids were important to the story. Joachim reminded DS that Richard Garner talked about cofree comonoids at a recent CT. Richard's HottEST talk was completely mind blowing for DS. Worlds collided.

Nelson joins David as undergraduate student researcher and very quickly taught David lots of cool stuff about comonoids and bimodules. Great synergy there. 

DS invented poly boxes, and everything started going faster---very fast-paced and relevant research while writing a book is not so good, so it got disjointed. The writing was going quickly, if not particularly coherent, but began to falter in mid-August when DS's hands began to hurt and DJM's teaching duties started. The pace crawled. 

DS personally hired Nelson as a typist. Nelson had been so helpful earlier, DS asks DJM if NN can be an author. DJM and DS realize the book would be better as two, DS peels off his part. 

DS asked NN if he was willing to join in the effort to make a quality product, and he was. They started in Jan 2021. They decided to teach a course about it at the Topos Institute in Jul 2021.

\section*{Acknowledgments}

Special thanks to David Jaz Myers: a brilliant colleague, a wonderful conversation partner, a congenial housemate, a superb chef, and an all-around good guy. 

Thanks go to John Baez, Eric Bond, Spencer Breiner, Kris Brown, Matteo Capucci, Valeria de Paiva, Joseph Dorta, Brendan Fong, Richard Garner, Bruno Gavranovi\'c, Neil Ghani, Ben Goertzel, Tim Hosgood, Samantha Jarvis, Max Lieblich, Owen Lynch, Joachim Kock, J\'er\'emie Koenig, Sophie Libkind, Joshua Meyers, Dominic Orchard, Evan Patterson, Brandon Shapiro, Juliet Szatko, Tish Tanski, Todd Trimble, Adam Vandervorst, Jonathan Weinberger, and Christian Williams.

\end{document}
