\documentclass[Book-Poly]{subfiles}
\begin{document}
%

\setcounter{chapter}{3}%Just finished 3.


%------------ Chapter ------------%
\chapter{Exercise solutions}
\footnotesize

\sol{exc.arenas}{
Consider the polynomial $p\coloneqq\2\yon^\3+\2\yon+\1$ and the associated arena.
\begin{enumerate}
	\item Draw the arena whose name is $p$ as a set of corollas.
	\item How many roots/positions does this arena have?
	\item How many decisions does this represent?
	\item For each corolla in the arena, say how many leaves it has.
	\item For each decision, how many options does it have?
\end{enumerate}
Now referring instead to the polynomial $q\coloneqq\yon^\nn+\4\yon$:
\begin{enumerate}[resume]
	\item Does the polynomial $q$ have a pure-power summand $\yon^\2$?
	\item Does the polynomial $q$ have a pure-power summand $\yon$?
	\item Does the polynomial $q$ have a pure-power summand $\4\yon$?
	\qedhere
\end{enumerate}
}
{
We are first considering the polynomial $p\coloneqq\2\yon^\3+\2\yon+\1$.
\begin{enumerate}
	\item Here is the arena $p$ drawn as a set of corollas (note that the order in which they are drawn does not matter):
	\[
	\begin{tikzpicture}[trees, sibling distance=3mm]
    \node (1) {$\bullet$} 
      child {}
      child {}
      child {};
    \node[right=.7 of 1] (2) {$\bullet$} 
      child {}
      child {}
      child {};
    \node[right=.5 of 2] (3) {$\bullet$} 
      child {};
    \node[right=.3 of 3] (4) {$\bullet$} 
      child {};
    \node[right=.3 of 4] (5) {$\bullet$};
  \end{tikzpicture}
  \]
	\item It has five (5) positions (which we here are calling roots).
	\item It represents five decisions, one per position.
	\item \label{sol.arenas.leaves} The first and second corollas have three leaves, the third and fourth corollas have one leaf, and the fifth corolla has no leaves.
	\item The set of options for each decision is the same as the set of leaves for each corolla, so just copy the answer from \cref{sol.arenas.leaves}. Sheesh! Who wrote this.
\end{enumerate}
Next we refer to the polynomial $q\coloneqq\yon^\nn+4\yon$.
\begin{enumerate}[resume]
	\item No, $q$ does not have $\yon^\2$ as a pure-power summand.
	\item Yes, $q$ does have $\yon$ as a pure-power summand.
	\item No, $q$ does not have $\4\yon$ as a pure-power summand, because $\4\yon$ is not a pure-power! But to make amends, we could say that $\4\yon$ is a summand; this means that there is some $q'$ such that $q=q'+\4\yon$. So $\3\yon$ is also a summand, but $\5\yon$ and $\yon^\2$ are not.
\end{enumerate}
}

\sol{exc.suitor_love}{
If you were a suitor choosing the arena you love, aesthetically speaking, which would strike your interest? Answer by circling the associated polynomial:
\begin{enumerate}
	\item $\yon^\2+\yon+\1$
	\item $\yon^\2+\3\yon^\2+\3\yon+\1$
	\item $\yon^\2$
	\item $\yon+\1$
	\item $(\nn\yon)^\nn$
	\item $S\yon^S$
	\item $\yon^{\1\0\0}+\yon^\2+\3\yon$
	\item Your poly's name $p$ here.
\end{enumerate}
Any reason for your choice? Draw a sketch of your arena.
}
{
Aesthetically speaking, here's a beautiful arena:
\[
\yon^\0+\yon^\1+\yon^\2+\yon^\3+\cdots
\]
It's reminiscent (and formally related) to the notion of lists: if $A$ is any set then $A^0+A^1+A^2+\cdots$ is the set of lists with entries in $A$. 

Here's a picture of this lovely arena:
\[
	\begin{tikzpicture}[trees, sibling distance=3mm]
    \node (1) {$\bullet$};
    \node[right=.3 of 1] (2) {$\bullet$}
      child {};
    \node[right=.4 of 2] (3) {$\bullet$} 
      child {}
      child {};
    \node[right=.6 of 3] (4) {$\bullet$} 
      child {}
      child {}
      child {};
    \node[right=.6 of 4] {$\cdots$};
  \end{tikzpicture}
\]
}

\sol{exc.changing_types}{
Think of another example where systems are sending each other information, but where the sort of information or who it's being sent to or received from can change based on the states of the systems involved. You might have more than two, say $\rr$-many, different wiring patterns in your situation.
}
{
When using an application, say a drawing program, there is often a menu at the top of the screen. The information I can send the system changes based on what menu I click. If no menu is clicked, I can interact in one way. When I click ``file" I can interact with the file system, saving or opening a file, and thus interacting with another ``part'' of the computer. The set of things I can do depends on context. 
}

\sol{exc.decision_streams}
{
\begin{enumerate}
	\item Draw a level-3 abbreviation of a decision stream of type $\yon^\2+\yon^\0$.
	\item Draw a level-4 abbreviation of a decision stream of type $\yon$.
	\item Draw a level-3 abbreviation of a decision stream of type $\nn\yon^\2$ by labeling every node with a natural number.
	\qedhere
\end{enumerate}
}
{
\begin{enumerate}
	\item Here's a level-3 abbreviation of a decision stream of type $\yon^\2+\yon^\0$.
\[
\begin{tikzpicture}[trees,
  level 1/.style={sibling distance=20mm},
  level 2/.style={sibling distance=10mm},
  level 3/.style={sibling distance=5mm},
  level 4/.style={sibling distance=2.5mm}]
  \node (a) {$\bullet$}
    child {node {$\bullet$}
    	child {node {$\bullet$}
    		child
    		child
    	}
    	child {node {$\bullet$}
  			}
    }
    child {node {$\bullet$}
    	child {node {$\bullet$}}
    	child {node {$\bullet$}
  		}
  	}
  ;
\end{tikzpicture}
\]
	\item Here's a level-4 abbreviation of a decision stream of type $\yon$.
\[
\begin{tikzpicture}[trees]
	\node (a) {$\bullet$}
		child {node {$\bullet$}
			child {node {$\bullet$}
				child {node {$\bullet$}
  				child
			}}};
\end{tikzpicture}
\]
	\item Here's a level-3 abbreviation of a decision stream of type $\nn\yon^\2$ where we indicate the position of each node by labeling it with a natural number.
\[
\begin{tikzpicture}[trees,
  level 1/.style={sibling distance=20mm},
  level 2/.style={sibling distance=10mm},
  level 3/.style={sibling distance=5mm},
  level 4/.style={sibling distance=2.5mm}]
  \node (a) {$27$}
    child {node {$5040$}
    	child {node {$192$}
    		child 
    		child
			}
    	child {node {$0$}
    		child 
    		child
  			}
    }
    child {node {$314159$}
    	child {node {$1000$}
   				child
  				child
  		}
			child {node {$1296$}
				child
				child
			}
  	}
  ;
\end{tikzpicture}
\]
\end{enumerate}
}

\sol{exc.representable_fun}
{
For each of the following functors $\smset\to\smset$, say if it is representable or not; if it is, say what the representing set is.
\begin{enumerate}
	\item The identity functor $X\mapsto X$.
	\item The constant functor $X\mapsto\2$.
	\item The constant functor $X\mapsto\1$.
	\item The constant functor $X\mapsto\0$.
	\item The functor $X\mapsto X^\nn$.
	\item The functor $X\mapsto 2^X$.
\qedhere
\end{enumerate}
}
{
Our goal is to say whether various functors are representable (of the form $X\mapsto\smset(S,X)$ for some $S$, called the representing set).
\begin{enumerate}
	\item The identity functor $X\mapsto X$ is represented by $S=\1$: a function $\1 \to X$ is just an element of $X$, so $\smset(\1,X) \iso X$.
	Alternatively, note that $X^1 \iso X$.
	\item \label{sol.representable_fun.2} The constant functor $X\mapsto\2$ is not representable: the functor sends $\1$ to $\2$, but $\1^S \iso \1 \not\iso \2$ for any set $S$.
	\item The constant functor $X\mapsto\1$ is representable by $S=\0$: there is exactly one function $\0 \to X$, so $\smset(\0,X) \iso \1$.
	Alternatively, note that $X^0 \iso \1$.
	\item The constant functor $X\mapsto\0$ is not representable, for the same reason as in \cref{sol.representable_fun.2}.
	\item The functor $X\mapsto X^\nn$ is represented by $S=\nn$, by definition.
	\item The functor $X\mapsto 2^X$ is not representable, for the same reason as in \cref{sol.representable_fun.2}.
\end{enumerate}
}




\end{document}
