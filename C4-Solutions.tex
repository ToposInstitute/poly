\documentclass[Book-Poly]{subfiles}
\begin{document}
%


\setcounter{chapter}{3}%Just finished 3.


%------------ Chapter ------------%
\chapter{Exercise solutions}

\sol{exc.arenas}{
Consider the polynomial $p\coloneqq\2\yon^\3+\2\yon+\1$ and the associated arena.
\begin{enumerate}
	\item Draw the arena whose name is $p$ as a set of corollas.
	\item How many roots/positions does this arena have?
	\item How many decisions does this represent?
	\item For each corolla in the arena, say how many leaves it has.
	\item For each decision, how many options does it have?
\end{enumerate}
Now referring instead to the polynomial $q\coloneqq\yon^\nn+\4\yon$:
\begin{enumerate}[resume]
	\item Does the polynomial $q$ have a pure-power summand $\yon^\2$?
	\item Does the polynomial $q$ have a pure-power summand $\yon$?
	\item Does the polynomial $q$ have a pure-power summand $\4\yon$?
	\qedhere
\end{enumerate}
}
{
We are first considering the polynomial $p\coloneqq\2\yon^\3+\2\yon+\1$.
\begin{enumerate}
	\item Here is the arena $p$ drawn as a set of corollas:
	\[
	\begin{tikzpicture}[trees, sibling distance=3mm]
    \node (1) {$\bullet$} 
      child {}
      child {}
      child {};
    \node[right=.7 of 1] (2) {$\bullet$} 
      child {}
      child {}
      child {};
    \node[right=.5 of 2] (3) {$\bullet$} 
      child {};
    \node[right=.3 of 3] (4) {$\bullet$} 
      child {};
    \node[right=.3 of 4] (5) {$\bullet$};
  \end{tikzpicture}
  \]
	\item It has five (5) positions (which we here are calling roots).
	\item It represents five decisions.
	\item The first and second corollas have three leaves, the third and fourth corollas have one leaf, and the fifth corolla has no leaves.
	\item The set of options for each decision is the same as the set of leaves for each corolla, so just copy the answer from \#4. Sheesh! Who wrote this.
\end{enumerate}
Next we refer to the polynomial $q\coloneqq\yon^\nn+4\yon$.
\begin{enumerate}[resume]
	\item No, $q$ does not have $\yon^\2$ as a pure-power summand.
	\item Yes, $q$ does have $\yon$ as a pure-power summand.
	\item No, $q$ does not have $\4\yon$ as a pure-power summand, because $\4\yon$ is not a pure-power! But to make amends, we could say that $\4\yon$ is a summand; this means that there is some $q'$ such that $q=q'+\4\yon$. So $\3\yon$ is also a summand, but $\5\yon$ and $\yon^\2$ are not.
\end{enumerate}
}



\end{document}
