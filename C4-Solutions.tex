\documentclass[Book-Poly]{subfiles}
\begin{document}
%

\setcounter{chapter}{3}%Just finished 3.


%------------ Chapter ------------%
\chapter{Exercise solutions}
\footnotesize

\sol{exc.arenas}{
Consider the polynomial $p\coloneqq\2\yon^\3+\2\yon+\1$ and the associated arena.
\begin{enumerate}
	\item Draw the arena whose name is $p$ as a set of corollas.
	\item How many roots/positions does this arena have?
	\item How many decisions does this represent?
	\item For each corolla in the arena, say how many leaves it has.
	\item For each decision, how many options does it have?
\end{enumerate}
Now referring instead to the polynomial $q\coloneqq\yon^\nn+\4\yon$:
\begin{enumerate}[resume]
	\item Does the polynomial $q$ have a pure-power summand $\yon^\2$?
	\item Does the polynomial $q$ have a pure-power summand $\yon$?
	\item Does the polynomial $q$ have a pure-power summand $\4\yon$?
	\qedhere
\end{enumerate}
}
{
We are first considering the polynomial $p\coloneqq\2\yon^\3+\2\yon+\1$.
\begin{enumerate}
	\item Here is the arena $p$ drawn as a set of corollas (note that the order in which they are drawn does not matter):
	\[
	\begin{tikzpicture}[trees, sibling distance=3mm]
    \node (1) {$\bullet$} 
      child {}
      child {}
      child {};
    \node[right=.7 of 1] (2) {$\bullet$} 
      child {}
      child {}
      child {};
    \node[right=.5 of 2] (3) {$\bullet$} 
      child {};
    \node[right=.3 of 3] (4) {$\bullet$} 
      child {};
    \node[right=.3 of 4] (5) {$\bullet$};
  \end{tikzpicture}
  \]
	\item It has five (5) positions (which we here are calling roots).
	\item It represents five decisions, one per position.
	\item \label{sol.arenas.leaves} The first and second corollas have three leaves, the third and fourth corollas have one leaf, and the fifth corolla has no leaves.
	\item The set of options for each decision is the same as the set of leaves for each corolla, so just copy the answer from \cref{sol.arenas.leaves}. Sheesh! Who wrote this.
\end{enumerate}
Next we refer to the polynomial $q\coloneqq\yon^\nn+4\yon$.
\begin{enumerate}[resume]
	\item No, $q$ does not have $\yon^\2$ as a pure-power summand.
	\item Yes, $q$ does have $\yon$ as a pure-power summand.
	\item No, $q$ does not have $\4\yon$ as a pure-power summand, because $\4\yon$ is not a pure-power! But to make amends, we could say that $\4\yon$ is a summand; this means that there is some $q'$ such that $q=q'+\4\yon$. So $\3\yon$ is also a summand, but $\5\yon$ and $\yon^\2$ are not.
\end{enumerate}
}

\sol{exc.suitor_love}{
If you were a suitor choosing the arena you love, aesthetically speaking, which would strike your interest? Answer by circling the associated polynomial:
\begin{enumerate}
	\item $\yon^\2+\yon+\1$
	\item $\yon^\2+\3\yon^\2+\3\yon+\1$
	\item $\yon^\2$
	\item $\yon+\1$
	\item $(\nn\yon)^\nn$
	\item $S\yon^S$
	\item $\yon^{\1\0\0}+\yon^\2+\3\yon$
	\item Your poly's name $p$ here.
\end{enumerate}
Any reason for your choice? Draw a sketch of your arena.
}
{
Aesthetically speaking, here's a beautiful arena:
\[
\yon^\0+\yon^\1+\yon^\2+\yon^\3+\cdots
\]
It's reminiscent (and formally related) to the notion of lists: if $A$ is any set then $A^0+A^1+A^2+\cdots$ is the set of lists with entries in $A$. 

Here's a picture of this lovely arena:
\[
	\begin{tikzpicture}[trees, sibling distance=3mm]
    \node (1) {$\bullet$};
    \node[right=.3 of 1] (2) {$\bullet$}
      child {};
    \node[right=.4 of 2] (3) {$\bullet$} 
      child {}
      child {};
    \node[right=.6 of 3] (4) {$\bullet$} 
      child {}
      child {}
      child {};
    \node[right=.6 of 4] {$\cdots$};
  \end{tikzpicture}
\]
}

\sol{exc.changing_types}{
Think of another example where systems are sending each other information, but where the sort of information or who it's being sent to or received from can change based on the states of the systems involved. You might have more than two, say $\rr$-many, different wiring patterns in your situation.
}
{
When using an application, say a drawing program, there is often a menu at the top of the screen. The information I can send the system changes based on what menu I click. If no menu is clicked, I can interact in one way. When I click ``file" I can interact with the file system, saving or opening a file, and thus interacting with another ``part'' of the computer. The set of things I can do depends on context. 
}

\sol{exc.decision_streams}
{
\begin{enumerate}
	\item Draw a level-3 abbreviation of a decision stream of type $\yon^\2+\yon^\0$.
	\item Draw a level-4 abbreviation of a decision stream of type $\yon$.
	\item Draw a level-3 abbreviation of a decision stream of type $\nn\yon^\2$ by labeling every node with a natural number.
	\qedhere
\end{enumerate}
}
{
\begin{enumerate}
	\item Here's a level-3 abbreviation of a decision stream of type $\yon^\2+\yon^\0$.
\[
\begin{tikzpicture}[trees,
  level 1/.style={sibling distance=20mm},
  level 2/.style={sibling distance=10mm},
  level 3/.style={sibling distance=5mm},
  level 4/.style={sibling distance=2.5mm}]
  \node (a) {$\bullet$}
    child {node {$\bullet$}
    	child {node {$\bullet$}
    		child
    		child
    	}
    	child {node {$\bullet$}
  			}
    }
    child {node {$\bullet$}
    	child {node {$\bullet$}}
    	child {node {$\bullet$}
  		}
  	}
  ;
\end{tikzpicture}
\]
	\item Here's a level-4 abbreviation of a decision stream of type $\yon$.
\[
\begin{tikzpicture}[trees]
	\node (a) {$\bullet$}
		child {node {$\bullet$}
			child {node {$\bullet$}
				child {node {$\bullet$}
  				child
			}}};
\end{tikzpicture}
\]
	\item Here's a level-3 abbreviation of a decision stream of type $\nn\yon^\2$ where we indicate the position of each node by labeling it with a natural number.
\[
\begin{tikzpicture}[trees,
  level 1/.style={sibling distance=20mm},
  level 2/.style={sibling distance=10mm},
  level 3/.style={sibling distance=5mm},
  level 4/.style={sibling distance=2.5mm}]
  \node (a) {$27$}
    child {node {$5040$}
    	child {node {$192$}
    		child 
    		child
			}
    	child {node {$0$}
    		child 
    		child
  			}
    }
    child {node {$314159$}
    	child {node {$1000$}
   				child
  				child
  		}
			child {node {$1296$}
				child
				child
			}
  	}
  ;
\end{tikzpicture}
\]
\end{enumerate}
}

\sol{exc.my_schema_and_tables}{
As above, we define the finitely presented category $\cat{C}$ according to \eqref{eqn.myschema} and the copresheaf $I$ on $\cat{C}$ according to \eqref{eqn.mytables}.
\begin{enumerate}
    \item What is $I(\text{Department})$?
    \item What is $I(\text{Admin})$?
    \item Composing Admin with FirstName yields a morphism from Department to String that we denote by Admin.FirstName.
    What is $I(\text{Admin.FirstName})$?
    \item Say we require that managers work in the same department as the employees they oversee.
    Write down an equation in $\cat{C}$ (like the one at the bottom of \eqref{eqn.myschema} that expresses this condition.
    \item How might we define $I(\text{String})$?
\end{enumerate}
}{
We refer to \eqref{eqn.myschema} and \eqref{eqn.mytables} to characterize the category $\cat{C}$ and the functor $I \colon \cat{C} \to \smset$.
\begin{enumerate}
    \item Here Department is an object of $\cat{C}$, so $I(\text{Department})$ is a set. From the Department column in the table on the right of \eqref{eqn.mytables}, we observe that $I(\text{Department}) = \{101, 102\}$.
    \item Here Admin is an morphism of $\cat{C}$ from Department to Employee, so $I(\text{Admin})$ is a function from $I(\text{Department}) = \{101, 102\}$ to $I(\text{Employee}) = \{1, 2, 3\}$. From the Admin column in the table on the right of \eqref{eqn.mytables}, we observe that $I(\text{Admin})(101) = 1$ and $I(\text{Admin})(102) = 3$.
    \item By functoriality, $I(\text{Admin.FirstName})$ is the function $I(\text{Admin})$ composed with $I(\text{FirstName})$.
    We have that $I(\text{Admin})$ sends $101$ to $1$ and $102$ to $3$, while the FirstName column tells us that $I(\text{FirstName})$ sends $1$ to Alan and $3$ to Carla.
    So $I(\text{Admin.FirstName})$ sends $101$ to Alan and $102$ to Carla.
    \item If every employee works in the same department as their manager, then Employee.WorksIn = Employee.Mngr.WorksIn.
    \item The name suggests that $I(\text{String})$ is the set of all possible strings of characters.
    Perhaps this could be defined as $\bigcup_{n \in \nn} A^n$, where $A$ is our alphabet of allowed characters, which may include the English letters, spaces, digits, and whatever other characters we allow.
    At the very least, since FirstName and Name are both morphisms to String, every entry in the FirstName and Name columns must be in $I(\text{String})$.
    So all we know for sure is that $\{\text{Alan, Ruth, Carla, Sales, IT}\} \subseteq I(\text{String})$.
\end{enumerate}
}

\sol{exc.representable_fun}
{
For each of the following functors $\smset\to\smset$, say if it is representable or not; if it is, say what the representing set is.
\begin{enumerate}
	\item The identity functor $X\mapsto X$, which sends each function to itself.
	\item The constant functor $X\mapsto\2$, which sends every function to the identity on $\2$.
	\item The constant functor $X\mapsto\1$, which sends every function to the identity on $\1$.
	\item The constant functor $X\mapsto\0$, which sends every function to the identity on $\0$.
	\item A functor $X\mapsto X^\nn$.
	If it were representable, where would it send each function?
	\item A functor $X\mapsto 2^X$.
	If it were representable, where would it send each function?
\qedhere
\end{enumerate}
}
{
Our goal is to say whether various functors are representable (of the form $X\mapsto\smset(S,X)$ for some $S$, called the representing set).
\begin{enumerate}
	\item The identity functor $X\mapsto X$ is represented by $S=\1$: a function $\1 \to X$ is just an element of $X$, so $\smset(\1,X) \iso X$.
	Alternatively, note that $X^1 \iso X$.
	\item \label{sol.representable_fun.2} The constant functor $X\mapsto\2$ is not representable: the functor sends $\1$ to $\2$, but $\1^S \iso \1 \not\iso \2$ for any set $S$.
	\item The constant functor $X\mapsto\1$ is representable by $S=\0$: there is exactly one function $\0 \to X$, so $\smset(\0,X) \iso \1$.
	Alternatively, note that $X^0 \iso \1$.
	\item The constant functor $X\mapsto\0$ is not representable, for the same reason as in \cref{sol.representable_fun.2}.
	\item The functor $\yon^\nn$ that sends $X\mapsto X^\nn$ is represented by $S=\nn$, by definition.
	It sends each function $h \colon X \to Y$ to the function $h^\nn \colon X^\nn \to Y^\nn$ that sends each $g \colon \nn \to X$ to $g \then h \colon \nn \to Y$.
	\item No $\smset \to \smset$ functor $X\mapsto \2^X$ is representable, for the same reason as in \cref{sol.representable_fun.2}.
	(There \emph{is}, however, a functor $\smset\op \to \smset$ sending $X \mapsto 2^X$ that is understood to be representable in a more general sense.)
\end{enumerate}
}



\sol{exc.representable_nt}
{
Prove that for any function $f\colon R\to S$, what we said was a natural transformation in \cref{prop.representable_nt} really is natural. That is, for any function $h\colon X\to Y$, show that the following diagram commutes:
\[
\begin{tikzcd}
	X^S\ar[r, "h^S"]\ar[d, "X^f"']&
	Y^S\ar[d, "Y^f"]\\
	X^R\ar[r, "h^R"']&
	Y^R\ar[ul, phantom, "?"]
\end{tikzcd}
\qedhere
\]
}
{
To show that
\[
\begin{tikzcd}[ampersand replacement=\&]
	X^S\ar[r, "h^S"]\ar[d, "X^f"']\&
	Y^S\ar[d, "Y^f"]\\
	X^R\ar[r, "h^R"']\&
	Y^R\ar[ul, phantom, "?"]\&
	\qedhere
\end{tikzcd}
\]
commutes, we note that by \cref{prop.representable_nt}, both vertical maps compose functions from $S$ with $f \colon R \to S$ from the left, and by \cref{def.representable}, both horizontal maps compose functions to $X$ with $h \colon X \to Y$ on the right.
So by the associativity of composition, the diagram commutes.
}


\sol{exc.representable_nt_components}
{
Let $X$ be an arbitrary set. For each of the following sets $R,S$ and functions $f\colon R\to S$, describe the $X$-component of, i.e.\ the function $X^S\to X^R$ coming from, the natural transformation $\yon^f\colon\yon^S\to\yon^R$.
\begin{enumerate}
	\item $R=\5$, $S=\5$, $f=\id$.  (Here you're supposed to give a function called $X^{\id_\5}\colon X^\5\to X^\5$.)
	\item $R=\2$, $S=\1$, $f$ is the unique function.
	\item $R=\1$, $S=\2$, $f(1)=1$.
	\item $R=\1$, $S=\2$, $f(1)=2$.
	\item $R=\0$, $S=\5$, $f$ is the unique function.
	\item $R=\nn$, $S=\nn$, $f(n)=n+1$.
\qedhere
\end{enumerate}
}
{
In each case, given $f \colon R \to S$, we can find the $X$-component $X^f \colon X^S \to X^R$ of the natural transformation $\yon^f\colon\yon^S\to\yon^R$ by applying \cref{prop.representable_nt}, which says that $X^f$ sends each $g \colon S \to X$ to $f \then g \colon R \to X$.
\begin{enumerate}
    \item If $R=\5$, $S=\5$, and $f=\id$, then $X^f$ is the identity function on $X^\5$.
    \item If $R=\2$, $S=\1$, and $f$ is the unique function, then $X^f$ sends each $g \in X$ (i.e. function $g \colon \1 \to X$) to the function that maps both elements of $\2$ to $g$.
    We can think of $X^f$ as the diagonal $X \to X \times X$.
	\item If $R=\1$, $S=\2$, and $f(1)=1$, then $X^f$ sends each $g \colon \2 \to X$ to $g(1)$, viewed as a function $\1 \to X$.
	We can think of $X^f$ as the left projection $X \times X \to X$.
	\item If $R=\1$, $S=\2$, and $f(1)=2$, then $X^f$ sends each $g \colon \2 \to X$ to $g(2)$, viewed as a function $\1 \to X$.
	We can think of $X^f$ as the right projection $X \times X \to X$.
	\item If $R=\0$, $S=\5$, and $f$ is the unique function, then $X^f$ is the unique function $X^\5 \to X^\0 \iso \1$.
	\item If $R=\nn$, $S=\nn$, and $f(n)=n+1$, then $X^f$ sends each $g \colon \nn \to X$ to the function $h \colon \nn \to X$ satisfying $h(n) = g(n+1)$ for all $n \in \nn$.
	We can think of $X^f$ as removing the first term of an infinite sequence of elements of $X$.
\end{enumerate}
}



\sol{exc.representable_nt_functorial}
{
Show that the construction in \cref{prop.representable_nt} is functorial
\begin{equation}
\yon^-\colon\smset\op\to\smset^\smset,
\end{equation}
as follows.
\begin{enumerate}
	\item Show that for any set $S$, we have $\yon^{\id_S}\colon\yon^S\to\yon^S$ is the identity.
	\item Show that for any functions $f\colon R\to S$ and $g\colon S\to T$, we have $\yon^g\then\yon^f=\yon^{f\then g}$.
\qedhere
\end{enumerate}
}
{
\begin{enumerate}
    \item The fact that $\yon^{\id_S}\colon\yon^S\to\yon^S$ is the identity is just a generalization of \cref{exc.representable_nt_components} \cref{exc.representable_nt_components.id}.
    For any set $X$, the $X$-component $X^{\id_S} \colon X^S \to X^S$ of $\yon^{\id_S}$ sends each $h \colon S \to X$ to $\id_S \then h = h$, so $X^{\id_S}$ is the identity on $X^S$.
    Hence $\yon^{\id_S}$ is the identity on $\yon^S$.
    \item Fix $f \colon R \to S$ and $g \colon S \to T$; we wish to show that $\yon^g \then \yon^f = \yon^{f \then g}$.
    It suffices to show component-wise that $X^g \then X^f = X^{f \then g}$ for every set $X$.
    Indeed, $X^g$ sends each $h \colon T \to X$ to $g \then h$; then $X^f$ sends $g \then h$ to $f \then g \then h = X^{f \then g}(h)$.
\end{enumerate}
}

\sol{exc.finish_proof_yoneda}
{
Whoever solves this exercise can say they've proved the Yoneda Lemma.
\begin{enumerate}
	\item Show that for any $a\in F(S)$, the maps $X^S\to F(X)$ given as in \cref{lemma.yoneda} are natural in $X$.
	\item Show that the two mappings from \cref{lemma.yoneda} are mutually inverse.
	\item Show that \eqref{eqn.yoneda} is natural in $F$.
	\item Show that \eqref{eqn.yoneda} is natural in $S$.
	\item As a corollary of \cref{lemma.yoneda}, show that $\yon^-\colon\smset\op\to\smset^\smset$ is fully faithful, in particular that there is an isomorphism $\nat(\yon^S,\yon^T)\cong S^T$.
\qedhere
\end{enumerate}
}
{
\begin{enumerate}
    \item Given $a \in F(S)$, naturality of the maps $X^S \to F(X)$ that send $g \colon S \to X$ to $F(g)(a)$ amounts to the commutativity of
    \[
    \begin{tikzcd}[ampersand replacement=\&]
    	X^S\ar[r, "h^S"]\ar[d, "F(-)(a)"']\&
    	Y^S\ar[d, "F(-)(a)"]\\
    	F(X)\ar[r, "F(h)"']\&
    	F(Y)
    \end{tikzcd}
    \]
    for all $h \colon X \to Y$.
    The top map $h^S$ sends any $g \colon X \to S$ to $g \then h$ (\cref{def.representable}), which is then sent to $F(g \then h)(a)$ by the right map.
    Meanwhile, the left map sends $g$ to $F(g)(a)$, which is then sent to $F(h)(F(g)(a))$ by the bottom map.
    So by the functoriality of $F$, the square commutes.
    
    \item First, we show that for any natural transformation $m \colon \yon^S \to F$, we have that $m^{m_S(\id_S)} = m$.
    Given a set $X$, the $X$-component of $m^{m_S(\id_S)}$ sends each $g \colon S \to X$ to $F(g)(m_S(\id_S))$; it suffices to show that this is also where the $X$-component of $m$ sends $g$.
    Indeed, by the naturality of $m$, the square
    \[
    \begin{tikzcd}[ampersand replacement=\&]
    	S^S\ar[r, "g^S"]\ar[d, "m_S"']\&
    	X^S\ar[d, "m_X"]\\
    	F(S)\ar[r, "F(g)"']\&
    	F(X)
    \end{tikzcd}
    \]
    commutes, so in particular
    \begin{equation} \label{eq.finish_proof_yoneda}
        F(g)(m_S(\id_S)) = m_X(g^S(\id_S)) = m_X(\id_S \then g) = m_X(g).
    \end{equation}
    In the other direction, we show that for any $a \in F(S)$, we have $m^a_S(\id_S) = a$: by construction, $m^a_S \colon S^S \to F(S)$ sends $\id_S$ to $F(\id_S)(a) = a$.
    
    \item Given functors $F, G \colon \smset^\smset$ and a natural transformation $\alpha \colon F \to G$, we wish to show that the naturality square
    \[
    \begin{tikzcd}[ampersand replacement=\&]
    	\nat(\yon^S,F)\ar[d, "- \then \alpha"']\ar[r, "\sim"]\&
    	F(S)\ar[d, "\alpha_S"]\\
    	\nat(\yon^S,G)\ar[r, "\sim"]\&
    	G(S)
    \end{tikzcd}
    \]
    commutes.
    The top map sends any $m \colon \yon^S \to F$ to $m_S(\id_S)$, which in turn is sent by the right map to $\alpha_S(m_S(\id_S)) = (m \then \alpha)_S(\id_S)$.
    This is also where the bottom map sends $m \then \alpha$, so the square commutes.
    
    \item Given a function $g \colon S \to X$, we wish to show that the naturality square on the left side of the diagram
    \[
    \begin{tikzcd}[ampersand replacement=\&]
    	\nat(\yon^S,F)\ar[d, "\yon^g \then -"']\ar[r, "\sim"]\&
    	F(S)\ar[d, "F(f)"]\\
    	\nat(\yon^X,F)\ar[r, "\sim"]\&
    	F(X)
    \end{tikzcd}
    \]
    commutes.
    The left map sends any $m \colon \yon^S \to F$ to $\yon^g \then m$, which is sent by the bottom map to $(\yon^g \then m)_X(\id_X) = m_X(X^g(\id_X)) = m_X(f \then \id_X) = m_X(g)$.
    Meanwhile, the top map sends $m$ to $m_S(\id_S)$, which is sent by the right map to $F(g)(m_S(\id_S))$.
    So the square commutes by \eqref{eq.finish_proof_yoneda}.
    
    \item Just take $F = \yon^T$ in \cref{lemma.yoneda}.
\end{enumerate}
}

\sol{exc.on_sums_prods_sets}
{
Let $I$ be a set and let $X_i=\1$ be a one-element set for each $i\in I$. 
\begin{enumerate}
	\item Show that there is an isomorphism of sets $I\cong\sum_{i\in I}\1$.
	\item Show that there is an isomorphism of sets $\1\cong\prod_{i\in I}\1$.
\end{enumerate}
As a special case, suppose $I\coloneqq\varnothing$ and $X\colon \varnothing\to\smset$ is the unique empty collection of sets.
\begin{enumerate}
	\item Is it true that $X_i=\1$ for each $i\in I$?
	\item Show that there is an isomorphism of sets $\0\cong\sum_{i\in\varnothing}X_i$.
	\item Show that there is an isomorphism of sets $\1\cong\prod_{i\in\varnothing}X_i$.
\qedhere
\end{enumerate}
}{
We are given a set $I$ and a dependent set $(X_i)_{i \in I}$ for which $X_i = \1$ for every $i \in I$.
\begin{enumerate}
    \item \label{sol.on_sums_prods_sets.sum}
    To show that $I \iso \sum_{i \in I} \1$, we note that $x \in \1 = \{1\}$ if and only if $x = 1$, so $\sum_{i \in I} \1 = \{(i, 1) \ | \ i \in I\}$.
    Then function $I \to \sum_{i \in I} \1$ that sends each $i \in I$ to $(i, 1)$ is clearly an isomorphism.
    
    \item \label{sol.on_sums_prods_sets.prod}
    To show that $\1 \iso \prod_{i \in I} \1$, it suffices to demonstrate that there is a unique dependent function $f \colon (i \in I) \to \1$.
    As $\1 = \{1\}$, such a function $f$ must always send $i \in I$ to $1$.
    This completely characterizes $f$, so there is only one such dependent function.
    
\end{enumerate}
Now $I = \varnothing$ and $X \colon \varnothing \to \smset$ is the unique empty collection of sets.
\begin{enumerate}[resume]
    \item \label{sol.on_sums_prods_sets.vac} Yes: since $I$ is empty, there are no $i \in I$.
    So it is true that $X_i = 1$ holds whenever $i \in I$ holds, because $i \in I$ never holds. 
    We say that this sort of statement is ``vacuously true.''
    
    \item As $I = \varnothing = \0$, we have $\0 = I \iso \sum_{i \in I} \1 = \sum_{i \in \varnothing} X_i$, where the middle isomorphism follows from \cref{sol.on_sums_prods_sets.sum} and the last equation follows from \cref{sol.on_sums_prods_sets.vac}.
    
    \item As $I = \varnothing = \0$, we have $\1 \iso \prod_{i \in I} \1 = \prod_{i \in \varnothing} X_i$, where the isomorphism on the left follows from \cref{sol.on_sums_prods_sets.prod} and the equation on the right follows from \cref{sol.on_sums_prods_sets.vac}.
\end{enumerate}
}


\sol{ex.dependent_product_as_sections}
{
Let $X_{(-)} : I \to \smset$ be a set depending on an $i \in I$. There is a
projection function
$\pi_1 : \sum_{i \in I} X_i \to I$
defined by $\pi_1(i, x) = i$.
\begin{enumerate}
\item What is the signature of the second projection $\pi_2(i, x) = x$?
  (Hint: its a dependent function).
\item A \emph{section} of a function $r : A \to B$ is a function $s : B \to
  A$ such that $r \circ s = \id_B$. Show that the dependent product is
  isomorphic to the set of sections of $\pi_1$:
  $$\prod_{i \in I} X_i \cong \{s : I \to \sum_{i \in I} X_i \mid \pi_1
  \circ s = \id_I\}.$$
\end{enumerate}
}{
We have a dependent set $X_{(-)} : I \to \smset$ and a projection function $\pi_1 : \sum_{i \in I} X_i \to I$ defined by $\pi_1(i, x) = i$.
\begin{enumerate}
    \item The second projection $\pi_2(i, x) = x$ sends each pair $p = (i, x) \in \sum_{i \in I} X_i$ to $x$, an element of $X_i$.
    Observing that we can write $i$ in terms of $p$ as $\pi_1(p)$, 
    
    \item 
\end{enumerate}
}

\end{document}
